\documentclass[a4paper,10pt]{article}

\usepackage{ucs}
\usepackage[utf8]{inputenc}
\usepackage[spanish]{babel}
\usepackage{fontenc}
\usepackage{graphicx}
\usepackage{enumerate}
\author{Roberto Castellor Morant}
\title{Ped 3 sistemas en tiempo real}
\date{\today}

\begin{document}
\maketitle


\section{Ejercicio 3}

Sobre los protocolos de acotación de la prioridad (priority ceiling protocols),
responder a las siguientes cuestiones:
\begin{enumerate}[a)]
	\item ¿qué cuestiones abordan los protocolos de acotación de la
		prioridad¿?
	\item ¿Qué forma toma el protocolo original de acotación de la prioridad?
	\item ¿Cómo se define el protocolo inmediato de acotación de la prioridad?
	\item Aunque el comportamiento en el peor de los casos de los dos 
		esquemas de acotación es idéntico (desde el punto de vista de la 
		planificación), existen algunas diferencias, indicar cuales son.

\end{enumerate}

\subsection{Respuesta a}

Los procolos de acotación de prioridad abordan la planificación de procesos
tratando de minimizar las situaciones de cadenas de bloqueo y eliminar
condiciones de fallo.

Existen dos tipos de protocolos:

\begin{itemize}
	\item Protocolo original de acotación de la prioridad
	\item Protocolo inmediato de acotación de la prioridad
\end{itemize}

Cuando se utiliza cualquiera de estos dos protocolos en un sistema
monoprocesador se cumple lo siguiente:

\begin{itemize}
	\item Un proceso de alta prioridad puede ser bloqueado por procesos de
		prioridad baja en una sola ocasión como máximo durante su
		ejecución
	\item Se previenen los bloqueos mutuos (interbloqueos)
	\item Se previenen los bloqueos transitivos
	\item Se aseguran los accesos mutualmente excluyentes a los recursos.
\end{itemize}

\subsection{Respuesta b}

El protocolo original toma la siguiente forma:

\begin{enumerate}
	\item Cada proceso tiene asignada una prioridad estática por defecto
	\item Cada recurso tiene definido un valor cota estático, que es la
		prioridad máxima de los procesos que lo están utilizando.
	\item Un proceso tiene una prioridad dinámica que es el máximo de su
		propia prioridad estática y de cualquiera de las que herede
		debido a que bloquea procesos de mayor prioridad.
	\item Un proceso sólo puede bloquear un recurso si su prioridad dinámica
		es mayor que la cota máxima de cualquier recurso actualmente
		bloqueado (excluyendo cualquiera que él mismo ya pudiera tener
		bloqueado).
\end{enumerate}

Se permite el bloqueo del primer recurso del sistema. El efecto del protocolo es
asegurar que el segundo recurso sólo pueda ser bloqueado si no existe un proceso
de mayor prioridad que utilice ambos recursos.

\subsection{Respuesta c}

El algoritmo inmediato toma un enfoque más sencillo, fija la prioridad de un
proceso tan pronto bloquea un recurso, el procolo se define como sigue:

\begin{enumerate}
	\item Cada proceso tiene asignada una prioridad por defecto
	\item Cada recurso tiene definido un valor cota estático, que es la
		prioridad máxima de los procesos que lo utilizan
	\item Un proceso tiene una prioridad dinámica, que es el máximo entre su
		propia prioridad estática y los valores techo de cualquier
		recurso que tenga bloqueado.
\end{enumerate}

Como consecuencia de esta última regla, un proceso sólo podrá ser bloqueado al
principio de su ejecución. Una vez comience a ejecutarse, todos los recursos
necesarios deberían estar libres, si no lo estuvieran algun proceso tendría una
prioridad mayor o igual y la ejecución del proceso deberá ser pospuesta.

\subsection{Respuesta d}

Existen estas diferencias:

\begin{enumerate}
	\item ICPP es más sencillo de implementar que el original(OCPP), no hay
		que monitorizar las relaciones de bloqueo
	\item ICPP produce menos cambios de contexto, el bloqueo es previo a la
		primera ejecución
	\item ICPP requiere más cambios de prioridad, estos se producen con
		todas las utilizaciones de recursos, OCPP modifica la prioridad
		solo si se producen bloqueos
\end{enumerate}

\end{document}
