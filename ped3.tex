\section{Ejercicio 3}

Sobre los protocolos de acotación de la prioridad (priority ceiling protocols),
responder a las siguientes cuestiones:
\begin{enumerate}[a)]
	\item ¿qué cuestiones abordan los protocolos de acotación de la
		prioridad¿?
	\item ¿Qué forma toma el protocolo original de acotación de la prioridad?
	\item ¿Cómo se define el protocolo inmediato de acotación de la prioridad?
	\item Aunque el comportamiento en el peor de los casos de los dos 
		esquemas de acotación es idéntico (desde el punto de vista de la 
		planificación), existen algunas diferencias, indicar cuales son.

\end{enumerate}

\subsection{Respuesta a}

Los procolos de acotación de prioridad abordan la planificación de procesos
tratando de minimizar las situaciones de cadenas de bloqueo y eliminar
condiciones de fallo.

Existen dos tipos de protocolos:

\begin{itemize}
	\item Protocolo original de acotación de la prioridad
	\item Protocolo inmediato de acotación de la prioridad
\end{itemize}

Cuando se utiliza cualquiera de estos dos protocolos en un sistema
monoprocesador se cumplen lo siguiente:

\begin{itemize}
	\item Un proceso de alta prioridad puede ser bloqueado por procesos de
		prioridad baja en una sola ocasión como máximo durante su
		ejecución
	\item Se previenen los bloqueos mutuos (interbloqueos)
	\item Se previenen los bloqueos transitivos
	\item Se aseguran los accesos mutualmente excluyentes a los recursos.
\end{itemize}

\subsection{Respuesta b}

\begin{enumerate}
	\item Cada proceso tiene asignada una prioridad estática por defecto
	\item Cada recurso tiene definido un valor cota estático, que es la
		prioridad máxima de los procesos que lo están utilizando.
	\item Un proceso tiene una prioridad dinámica que es el máximo de su
		propia prioridad estática y de cualquiera de las que herede
		debido a que bloquea procesos de mayor prioridad.
	\item Un proceso sólo puede bloquear un recurso si su prioridad dinámica
		es mayor que la cota máxima de cualquier recurso actualmente
		bloqueado (excluyendo cualquiera que él mismo ya pudiera tener
		bloqueado).
\end{enumerate}

\subsection{Respuesta c}

\subsection{Respuesta d}

