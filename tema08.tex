\section{Control de recursos}

Como continuación del tema anterior, en este tema se tratan los procesos
competitivos. Un tema importante en este caso es la distinción entre
sincronización condicional y sincronización evitable en el modelo de
concurrencia.


\subsection{Interbloqueo o bloqueo mutuo}

Tambien conocido como deadlock se trata del bloqueo permanente de un conjunto de
procesos o hilos de ejecucion que compiten o se comunican entre ellos en un
sistema concurrente. No existe una solución general para los interbloqueos.

Es posible representar bloqueos mutuos mediante grafos dirigidos, el proceso es
representado por un cuadrado y el recurso por un circulo. Cuando un proceso
solicita un recurso se dibuja una flecha dirigida desde el el proceso al
recurso, cuando el recurso esta asignado a un proceso la flecha esta dirigida
desde el recurso al proceso.

Existen cuatro condiciones necesarias para que se de un interbloqueo, estas son:

\begin{itemize}
	\item Condición de exclusión mutua, existe un recurso compartido por los
procesos al que solo puede acceder un proceso simultaneamente.
	\item Condición de retención y espera, al menos un proceso P ha
adquirido un recurso R y lo retiene mientras espera un recurso R2 que ha sido
asignado a otro proceso.
	\item Condición de no expropiación, los recursos no pueden ser
expropiados por otros procesos, tienen que ser liberados por sus propietarios.
	\item Condición de espera circular, dado un conjunto de proceso P0..PM,
P0 esta esperando un recurso adquirido por P1, que esta esperando un recurso
adquirido por P2, ..., que esta esperando un recurso adquirido por PM, que esta
esperando un recurso adquirido por P0. Esta condición implica la condición de
retención y espera.
\end{itemize}

