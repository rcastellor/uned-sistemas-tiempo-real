\section{Ejercicio 1}

Para la programación de las N-versiones (N-versión programming), contestar a las
siguientes preguntas:

\begin{enumerate}[a)]
	\item Explicar en que suposición se basa, cuando deja de ser válida y
		que hay que realizar para que la suposición sea cierta.
	\item ¿De qué es responsable el proceso director (driver process)?
	\item ¿Qué se entiende por votación inexacta (inexact voting)? Como se puede
	 resolver y en que consiste el problema de la comparación consistente
(consistent comparison problema).
	\item Enumerar y comentar los tres factores (aspectos= issues) principales del
éxito de la programación de las N-versiones.
\end{enumerate}

\subsection{Respuesta a}

La programación de N-versiones se basa en la suposición de que se puede
especificar de forma consistente y sin ambigüedad el funcionamiento de un
programa y los programas que se han desarrollado de forma independiente fallaran
también de forma independiente. Esta suposición deja de ser cierta si se utiliza
para desarrollar los distintos programas el mismo lenguaje y entorno de
programación ya que pueden ocurrir fallos relacionados con el mismo que no serán
independientes.

\subsection{Respuesta b}

El proceso director es el responsable de:
\begin{itemize}
	\item Invocar a cada una de las versiones.
	\item Esperar que las versiones realicen su trabajo.
	\item Comparar los resultados y actuar basándose en los mismos.
\end{itemize}

\subsection{Respuesta c}

En los votos que incluyan números reales sera difícil que los resultados de las
M-versiones sean exactos. Esto podría deberse a temas hardware o a la
sensibilidad de los algoritmos utilizados. Las técnicas utilizadas para la
comparación de estos resultados se denominan votación inexacta. El problema de
la comparación consistente se da cuando una aplicación tiene que realizar una
comparación basada en un valor finito dado en la especificación, el resultado de
la comparación determina el curso de la acción.

\subsection{Respuesta d}

\begin{enumerate}[a)]
	\item Especificación inicial. La gran mayoría de fallos en el software
		provienen de una especificación errónea. Un error de
		especificación se manifestará en todas las N-Versiones.
	\item Independencia del diseño. Cuando una especificación es compleja se
		producirá una distorsión de la comprensión de los requisitos
		entre los equipos independientes. Si estos requisitos se
		refieren a datos poco frecuentes puede ser que estos errores no
		se capturen en la fase de pruebas.
	\item Presupuesto adecuado. Un sistema de N-versiones aumentara mucho el
		presupuesto y supondrá problemas de mantenimiento. Además no
		esta claro que no se pueda producir un sistema más fiable si
		todos los recursos para construir las N-versiones son utilizados
		para desarrollar una única versión.
\end{enumerate}
