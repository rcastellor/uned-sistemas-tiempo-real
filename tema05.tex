\section{Sincronización y comunicación basada en variables compartidas}

Este tema, junto con el siguiente, se centra en el estudio de la comunicación
entre tareas. En concreto en este primer tema se escriben los métodos de
variables compartidas, incluyendo la utilización de semáforos, monitories,
variables compartidas y objetos protegidos.


El comportamiento correcto de un programa concurrente depende estrechamente de
la sincronización y la comunicación entre procesos. Sincronizar es satisfacer
las restricciones en el entrelazado de las acciones de diferentes procesos.
Algunas formas de comunicación requieren sincronización y la sincronización
puede ser considerada como comunicación sin contenido.


La comunicación entre procesos se basa normalmente o en el uso de variables
compartidas o en el paso de mensajes. Las variables compartidas son objetos a
los que puede acceder mas de un proceso, la comunicación puede realizarse
referenciando en cada proceso dichas variables cuando sea apropiado. El paso de
mensajes implica el intercambio explícito de datos entre dos procesos mediante
un mensaje que pasa de un proceso a otro siguiendo algún mecanismo. Hay que
señalar que la elección entre variables compartidas y paso de mensajes deben
realizarla los diseñadores de los lenguajes o de los sistemas operativos y no
implica que deba utilizarse un método particular de implementación. Las
variables compartidas son más fáciles de soportar si hay memoria compartida
entre procesos, pero pueden ser utilizadas incluso si el hardware incorpora un
medio de comunicación. Este tema se centrará en las primitivas de sincronización
y comunicación basadas en memoria compartida. En particular se verán los
conceptos de espera ocupada, semáforos, regiones criticas condicionales,
monitores, tipos protegidos y métodos sincronizados.


Una secuencia de sentencias que deben aparecer como ejecutada indivisiblemente
se denomina \emph{sección crítica}. La sincronización que se precisa para
proteger una sección crítica se conoce como \emph{exclusión mutua}. La exclusión
mutua no es la única sincronización importante. La sincronización condicionada,
o de condición, es otro requisito significativo, y es necesaria cuando un
proceso desea realizar una operación que sólo puede ser realizada adecuadamente,
o de forma segura, si otro proceso ha realizado alguna acción o está en algún
estado definido.

\subsection{Espera ocupada}

Una forma de implementar la sincronización es comprobar las variables
compartidas que actuan como indicadores en un conjunto de procesos. Esta
aproximación sirve razonablemente bien para implementar sincronización de
condición, pero no hay un método simple para la exclusión mutua. Para indicar
una condición, un proceso activa el valor de un indicador; para esperar por esta
condición, otro proceso comprueba este indicador y sólo continúa cuando se lee
el valor apropiado:

\begin{lstlisting}
process P1; (* proceso esperando *)
	...
	while indicador = abajo do
		null
	end;
	...
end P1;

process P2; (* proceso indicando *)
	...
	indicador := arriba;
	...
end P2;
\end{lstlisting}

Si la condición no es aún correcta, P1 no tiene elección y deberá continuar en
el bucle para volver a comprobar el indicador. Esto se conoce como \emph{espera
ocupada}, y también como \emph{giro} -y a los indicadores como \emph{cerrojos de
giro} (spin locks)-.

Los algoritmos de espera presentan algunas dificultades que pueden resumirse
como:

\begin{iterate}
\item Los protocolos de espera ocupada son difíciles de diseñar y comprender, y
	es complicado probar su corrección.
\item Los programas de prueba pueden ignorar entrelazamientos raros que rompen
	la exclusión mutua o llevan a un interbloqueo activo.
\item Los bucles de espera ocupada son ineficientes.
\item Una tarea no fiable que utilice falsamente las variables compartidas,
	corromperá el sistema completo
\end{iterate}

Ningún lenguaje de programación concurrente se basa completamente en espera
ocupada y variables compartidas, hay otros métodos y primitivas, los semáforos y
monitores se describen en las siguientes secciones.
