\section{Programación concurrente}

En este tema se introuce la noción de proceso, tarea y hebra o hilo y revisa los
modelos que utilizan los diseñadores de lenguajes y de sistemas operativos. El
término tarea se utiliza genéricamente para representar una actividad
concurrente. Se aprovecha también este tema pra estudiar la distribución de
tareas cuando se dispone de multiprocesadores o de sistemas distribuidos.
Dejándose para los siguientes dos temas la comunicación entre tareas.
