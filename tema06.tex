\section{Sincronización y comunicación basada en mensajes}

Este tema es la continuación del anterior y resalta la importancia que tienen en
los lenguajes modernos los métodos basados en mensajes para la comunicación y
sincronización.

La sincronización basada en mensajes se basa en tres fases y/o clasificaciones:

\begin{enumerate}
	\item Modelo de sincronización: las variaciones en el modelo de
sincronización dependen de la semántica de envío.

	\begin{itemize}
		\item Envío asíncrono, el emisor envía un mensaje y continua
inmediatamente con sus acciones después de enviarlo. Requiere el uso de buffers.
		\item Envío síncrono, el emisor envía el mensaje y no continua
hasta que recibe el acuse de recibo del receptor.
		\item Invocación remota, el emisor espera a la elaboración de la
respuesta por parte del receptor. La espera es más larga que en el envío
síncrono.
	\end{itemize}

	\item Nombrado de procesos:
		\begin{itemize}
			\item Nombrado directo: se hace referencia explicita al
receptor
			\item Nombrado indirecto: se usa una entidad intermedia
para el envío (canal, tubería, etc...)
			\item Nombrado simétrico: el emisor y el receptor se
envían en mensajes entre sí. Especificado en todo momento los nombres.
			\item Nombrado asimétrico, no se nombra la fuente
especifica de recepción (Cliente/Servidor)
		\end{itemize}
	\item Estructura de mensajes: algunos lenguajes de programación
concurrentes restringen la estructura de sus mensajes.
\end{enumerate}

En la sincronización los procesos deben estar preparados para la comunicación,
si uno esta preparado el otro deberá esperar a que lo esté.

Espera selectiva, el receptor(servidor) tiene que esperar a que quede libre el
canal de comunicación. Podemos permitir esperar a varios procesos realizando una
selección en el servidor, mediante el uso de guardas se elige el procedimiento
que se ejecuta en el servidor.

Se puede dar el caso de que varias guardas se evalúen como verdadera, en este
caso la ejecución será indeterminista, es decir, se elegirá la opción
aleatoriamente.

Se produce una cita cuando un cliente invoca a un servicio de un servidor y este
acepta.

