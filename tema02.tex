\section{Fiabilidad y tolerancia a fallos}

Este tema, junto con el tema 3, trata de la producción de componentes de
software fiables. Aunque se realizan consideraciones sobre la prevención de
fallos, la atención se dedica principalmente a la tolerancia a fallos. Se
condirean las técnicas de recuperación de errores hacia adelante y hacia atrás.
El manejo de excepciones se estudia en el siguiente tema.

Causas que pueden propiciar el fallo de un sistema de tiempo real:
\begin{enumerate}
\item Especificación inadecuada.
\item Defectos provocados por errores de diseño.
\item Defectos provocados por fallos en componentes del procesador.
\item Defectos provocados por interferencias transitorias o permanentes en el
subsistema de comunicaciones.
\end{enumerate}

Los errores relacionados con el diseño o la especificación son dificiles de
preveer mientras que los errores provocados por fallos son en cierto modo
predecibles. Los lenguajes de programación de tiempo real tienen que ser
altamente fiables.

\subsection{Fiabilidad, fallos y defectos}

Fiabilidad: una medida del éxito con el que el sistema se ajusta a alguna
especificación definitiva de su comportamiento.

Fallo del sistema: Cuando el comportamiento de un sistema se desvía del
especificado para él, se dice que es un fallo.

Se pueden distinguir tres tipos de fallos:

\begin{enumerate}
	\item Fallos transitorios, Un fallo transitorio comienza en un instante
		de tiempo concreto, se mantiene en el sistema durante algún
		periodo, y luego desaparece. Ejemplos de este tipo de fallos se
		dan en componentes hardware en los que se produce una reacción
		adversa a una interferencia externa, como la producida por un
		campo eléctrico o por radioactividad. Después de que la
		perturbación desaparece, lo hace también el fallo (aunque no
		necesariamente el error inducido). Muchos de los fallos en los
		sistemas de comunicación son transitorios.
	\item Fallos permanentes Los fallos permanentes comienzan en un instante
		determinado y permanecen en el sistema hasta que son reparados;
		es el caso, por ejemplo, de un cable roto o de un error de
		diseño de software.
	\item Fallos intermitentes Son fallos transitorios que ocurren de vez en
		cuando. Un ejemplo es un componente hardware sensible al calor,
		que funciona durante un rato, deja de funcionar, se enfría, y
		entonces comienza a funcionar de nuevo
\end{enumerate}

\subsection{Modos de fallo}

Se pueden identificar dos dominios generales de modos de fallo:

\begin{enumerate}
	\item Fallos de valor, el valor asociado con el servicio es erróneo.
	\item Fallos de tiempo, el servicio se completa a destiempo.
\end{enumerate}

Las combinaciones de fallos de valor y de tiempo se denominan fallos
arbitrarios.

Un fallo de valor fuera del rango esperado para el servicio se denomina error de
límites, son fallos fácilmente reconocibles.

Los fallos en el dominio del tiempo se pueden englobar en:

\begin{enumerate}
	\item Demasiado pronto, el servicio es entregado antes de lo requerido.
	\item Demasiado tarde, el servicio se entrega después de lo requerido,
		se puede hablar de error de prestaciones.
	\item Infinitamente tarde, el servicio nunca es entregado, fallo de
		omisión.
\end{enumerate}

Esta clasificación se puede ampliar con fallos de encargo o improvisación cuando
el servicio es entregado sin ser esperado.

Dada la clasificación de fallos se puede definir algunas suposiciones respecto
al modo en que los sistemas pueden fallar:

\begin{enumerate}
	\item Fallo descontrolado, un sistema que produce fallos arbitrarios
		tanto en el dominio del valor como del tiempo
	\item Fallo de retraso, un sistema produce servicios correctos en el
		dominio del valor pero no en el del tiempo.
	\item Fallo de silencio, cuando el sistema falla bruscamente sin haber
		tenido fallos de valor o de tiempo, a partir del fallo todos los
		servicios subsiguientes también sufren fallos de omisión.
	\item Fallo de parada, un sistema que cumple los requisitos de un fallo
		de silencio pero permite que otros sistemas detectan que ha
		entrado en el estado de fallo
	\item Fallo controlado, un sistema falla de una forma especificada y
		controlada.
	\item Sin fallos, un sistema produce los servicios correctos, tanto en
		el dominio del valor como del tiempo.
\end{enumerate}

\subsection{Estrategias para sistemas fiables}

Para construir sistemas mas fiables podemos destacar la prevención de fallos
limitando la introducción de fallos en la construcción del sistema con:

\begin{itemize}
	\item Mejor hardware e interconexión de los componentes
	\item Aislar el hardware para la prevención de interferencias
		ambientales
	\item Especificación rigurosa del software con métodos de programación
		orientada a objetos
	\item Uso de herramientas especializadas del diseño del software
\end{itemize}

Tambén podemos enfocarnos en la eliminación de fallos aunque a pesar de las
pruebas realizadas para descubrir fallos nunca podemos demostrar su ausencia,
las pruebas no pueden ser exhaustivas.

\subsection{Nivel de tolerancia a fallos}

El nivel de tolerancia a fallos lo podemos describir con este esquema:

\begin{itemize}
	\item Tolerancia total: el sistema continua funcionando después de un
		fallo por un tiempo limitado, en este tiempo no se afecta ni la
		funcionalidad ni el rendimiento del sistema.
	\item Caída suave: el sistema sigue funcionando después de un fallo pero
		con aspectos de funcionalidad deteriorados
	\item Fallo seguro: se establece una parada segura que no produce
		efectos en la tolerancia respecto a otros componentes del
		sistema
\end{itemize}

Las técnicas de tolerancia a fallos se basan en la adición al sistema de
elementos que permiten la recuperación, estos elementos se llaman redundantes en
el sentido de que su uso no es necesario en caso del funcionamiento normal del
sistema.

Respecto al hardware existen dos estrategias de tolerancia a fallos:

\begin{itemize}
	\item Estática (enmascaramiento): módulos redundantes hardware que
		realizan la misma función, el sistema debe comparar los
		diferentes resultados para ser capaz de desechar los resultados
		incorrectos.
	\item Dinámica, la redundancia se produce dentro del propio componente,
		el mismo componente informa del error en la salida.
\end{itemize}

También podemos hablar de tolerancia a fallos en el software:

\begin{itemize}
	\item Estático, Programación N-versiones, se trata de introducir
		diferentes componentes software que sean capaces de producir
		cálculos alternativos de la misma función y proporcionar
		resultados a comparar para asegurar que están libres de errores.

		Los diferentes modelos deberían ser programados por personas,
		lenguajes y compiladores diferenes.
		
		Los programas son ejecutados por un programa director que los
		ejecuta concurrentemente y serán comparados por el mismo. Los
		resultados (votos) deben ser idénticos, si alguno es diferente
		este resultado será considerado como erróneo.

		Un problema que puede aparecer en los sistemas de medición de
		valores reales es la precisión de estos valores, necesitaremos
		entonces técnicas de comparación inexactas.

		El éxito de la programación de N-versiones depende de:

		\begin{itemize}
			\item Correcta especificación inicial: la especificación
				del diseño provocará que los diferentes
				componentes software funcionen de forma
				correcta.
			\item Independiencia en el diseño: es importante que se
				utilicen diferentes personas y herramientas para
				la implementación de los diferentes componentes.
			\item Presupuesto adecuado para desarrollar las tres
				versiones
		\end{itemize}
	\item Redundancia dinámica

		El error es informado por el propio módulo, que debe detectarlo.
		La detección puede ser por el entorno o por la apliación.

		El módulo debe ser capaz de confinar el error y valorar sus
		daños. El confinamiento consiste en establecer cortafuegos que
		delimiten el alcance del efecto de los errores. Se pueden
		realizar por:
		\begin{itemize}
			\item descomposición modular: delimita la difusión del
				error en el módulo.
			\item Acciones atómica: en caso de error devuelve el
				sistema a un estado de consistencia
		\end{itemize}

		Recuperación de errores, hay que tratar de volver al estado de
		operación normal o de degradación en la funcionalidad. Existen
		dos tipos de recuperación:

		\begin{itemize}
			\item Recuperación hacia adelante, se continua con la
				ejecución del sistema haciendo correcciones
				selectivas de su estado
			\item Recuperación de errores hacia atrás, en caso de
				error el sistema se restaura a un estado seguro
				previo y se ejecuta desde ese punto - checkpoint

				Una ventaja de la recuperación de errores hacia
				atrás es que el sistema no necesita localizar el
				error.

				Como inconvenientes:

				\begin{itemize}
					\item No se puede recuperar de errores
						del entorno
					\item Efecto dominó, se retorna a puntos
						de recuperación debido a
						posibles datos incorrectos en la
						comunicación entre varios
						procesos.
					\item El error puede volver a ocurrir,
						debe ser eliminado en dos
						etapas: localización y
						reparación.
				\end{itemize}
		\end{itemize}
	\item Bloques de recuperación

		Se trata de bloques normales de programación con un punto de
		recuperación al principio y un test de aceptación al final que
		comprueba que el sistema se encuentra en un estado aceptable al
		finalizar la ejecución del bloque. Si existe un fallo se
		restaura el estado de ejecución al inicio del bloque continuando
		con un procedimiento alternativo. Si todas las alternativas
		presentan error el bloque mostrará un estado de error.

		Es el test de aceptación el que se encarga de la detección de el
		error lo que puede suponer una sobrecarga. Por eso se introduce
		cierto margen de tolerancia para la aceptación.
\end{itemize}

