\section{Fiabilidad y tolerancia a fallos}

Este tema, junto con el tema 3, trata de la producción de componentes de
software fiables. Aunque se realizan consideraciones sobre la prevención de
fallos, la atención se dedica principalmente a la tolerancia a fallos. Se
condirean las técnicas de recuperación de errores hacia adelante y hacia atrás.
El manejo de excepciones se estudia en el siguiente tema.

Causas que pueden propiciar el fallo de un sistema de tiempo real:
\begin{enumerate}
\item Especificación inadecuada.
\item Defectos provocados por errores de diseño.
\item Defectos provocados por fallos en componentes del procesador.
\item Defectos provocados por interferencias transitorias o permanentes en el
subsistema de comunicaciones.
\end{enumerate}

Los errores relacionados con el diseño o la especificación son dificiles de
preveer mientras que los errores provocados por fallos son en cierto modo
predecibles. Los lenguajes de programación de tiempo real tienen que ser
altamente fiables.

\subsection{Fiabilidad, fallos y defectos}

Fiabilidad: una medida del éxito con el que el sistema se ajusta a alguna
especificación definitiva de su comportamiento.

Fallo del sistema: Cuando el comportamiento de un sistema se desvía del
especificado para él, se dice que es un fallo.

Se pueden distinguir tres tipos de fallos:

\begin{enumerate}
	\item Fallos transitorios, Un fallo transitorio comienza en un instante
		de tiempo concreto, se mantiene en el sistema durante algún
		periodo, y luego desaparece. Ejemplos de este tipo de fallos se
		dan en componentes hardware en los que se produce una reacción
		adversa a una interferencia externa, como la producida por un
		campo eléctrico o por radioactividad. Después de que la
		perturbación desaparece, lo hace también el fallo (aunque no
		necesariamente el error inducido). Muchos de los fallos en los
		sistemas de comunicación son transitorios.
	\item Fallos permanentes Los fallos permanentes comienzan en un instante
		determinado y permanecen en el sistema hasta que son reparados;
		es el caso, por ejemplo, de un cable roto o de un error de
		diseño de software.
	\item Fallos intermitentes Son fallos transitorios que ocurren de vez en
		cuando. Un ejemplo es un componente hardware sensible al calor,
		que funciona durante un rato, deja de funcionar, se enfría, y
		entonces comienza a funcionar de nuevo
\end{enumerate}

\subsection{Modos de fallo}

Se pueden identificar dos dominios generales de modos de fallo:

\begin{enumerate}
	\item Fallos de valor, el valor asociado con el servicio es erróneo.
	\item Fallos de tiempo, el servicio se completa a destiempo.
\end{enumerate}

Las combinaciones de fallos de valor y de tiempo se denominan fallos
arbitrarios.

Un fallo de valor fuera del rango esperado para el servicio se denomina error de
límites, son fallos fácilmente reconocibles.

Los fallos en el dominio del tiempo se pueden englobar en:

\begin{enumerate}
	\item Demasiado pronto, el servicio es entregado antes de lo requerido.
	\item Demasiado tarde, el servicio se entrega después de lo requerido,
		se puede hablar de error de prestaciones.
	\item Infinitamente tarde, el servicio nunca es entregado, fallo de
		omisión.
\end{enumerate}

Esta clasificación se puede ampliar con fallos de encargo o improvisación cuando
el servicio es entregado sin ser esperado.

Dada la clasificación de fallos se puede definir algunas suposiciones respecto
al modo en que los sistemas pueden fallar:

\begin{enumerate}
	\item Fallo descontrolado, un sistema que produce fallos arbitrarios
		tanto en el dominio del valor como del tiempo
	\item Fallo de retraso, un sistema produce servicios correctos en el
		dominio del valor pero no en el del tiempo.
	\item Fallo de silencio, cuando el sistema falla bruscamente sin haber
		tenido fallos de valor o de tiempo, a partir del fallo todos los
		servicios subsiguientes también sufren fallos de omisión.
	\item Fallo de parada, un sistema que cumple los requisitos de un fallo
		de silencio pero permite que otros sistemas detectan que ha
		entrado en el estado de fallo
	\item Fallo controlado, un sistema falla de una forma especificada y
		controlada.
	\item Sin fallos, un sistema produce los servicios correctos, tanto en
		el dominio del valor como del tiempo.
\end{enumerate}
