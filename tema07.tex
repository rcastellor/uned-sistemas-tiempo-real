\section{Acciones atómicas, Tareas Concurrentes y Fiabilidad}

En este tema se amplian las discusiones iniciales sobre tolerancia a fallos
describiendo con cuánta fiabilidad puede ser programada la cooperación entre
procesos. Para esta discusión es fundamental la noción de acción atómica y las
técnicas de manejo de eventos asíncronos.

Los objetivos del tema pasan por:

\subsection{Acciones atómicas}

Una acción es atómica si las tareas que la llevan a cabo no son conscientes de
las actividades de las primeras, durante el tiempo en el que llevan a cabo la
acción atómica.

\begin{itemize}
	\item Sin comunicación con otras tareas
	\item No detectan cambios de estado externos
	\item Los cambios de estado internos no se comunican hasta finalizar
\end{itemize}

Las acciones atómicas son vistas desde fuera como indivisibles e instantáneas.

Para permitir la descomposición modular de las acciones atomicas se incorpora la
noción de acciones atómicas anidadas. Los procesos comprometidos en una acción
anidada deben ser un subconjunto de los que están involucrados en el nivel
externo de la acción.

Las propiedades que tiene una acción atómica:

\begin{itemize}
	\item Una acción es atómica si los procesos que la realizan no saben de
		la existencia de ningún otro proceso activo, y ningún otro
		proceso activo tiene constancia de las actividades de los
		procesos durante el tiempo que en el que están realizando la
		acción.
	\item Una acción es atómica si los procesos que la realizan no se
		comunican con otros procesos mientras está siendo realizada la
		acción.
	\item Una acción es atómica si los procesos que la realizan no pueden
		detectar ningún cambio de estado salvo aquellos realizados por
		ellos mismos, y si no revelan sus cambios de estado hasta que la
		acción se haya completado.
	\item Las acciones son atómicas si, en lo que respecta a otros procesos,
		pueden ser consideradas indivisibles e instantáneas, de forma
		que los efectos sobre el sistema sean como si estuvieran
		entrelazadas y no en concurrencia.
\end{itemize}

\subsection{Acciones atómicas recuperables}

La expresión transacciones atómicas se ha utilizado frecuentemente en el marco
conceptual de los sistemas operativos y las bases de datos. Una transacción
atómica tiene todas las propiedades de una acción atómica, más la
característica adicional de que su ejecución puede tener exito o fallar. Por
fallo se entiende la ocurrencia de un error del que la transacción no puede
recuperarse, por ejemplo, un fallo de procesador. Si falla una unidad atómica,
los componentes del sistema que están siendo manipulados por la acción pueden
quedar en estado inconsistente. Ante un fallo, una transacción atómica
garantiza que los componentes son devueltos a su estado original. Las
transacciones atómicas a veces se conocen como acciones recuperables.

Las dos propiedades distintivas de las transacciones atómicas son:

\begin{itemize}
	\item Atomicidad de fallo, lo que significa que la transacción debe o
		bien ser completada con éxito o (en el caso de fallar) no tener
		efecto.
	\item Atomicidad de sincronización (o aislamiento), lo que significa
		que la transacción es indivisible, en el sentido de que su
		ejecución parcial no puede ser observada por ninguna transacción
		que se esté ejecutando concurrentemente.
\end{itemize}

\subsection{Notificación asíncrona}

El requisito fundamental de un mecanismo de notificación asíncrona es permitir
que un proceso responda rápidamente a una condición que ha sido detectada por
otro proceso. El énfasis esta en la prontitud de la respuesta, obviamente, un
proceso siempre puede responder a un evento simplemente sondeando o esperando
ese evento. La notificación del evento puede construirse sobre el mecanismo de
comunicación y sincronización de proceso. El proceso, cuando está disponible
para recibir el evento, simplemente realiza la petición apropiada.

Desafortunadamente, hay ocasiones en las que no es adecuado esperar o sondear la
ocurrencia de enventos, como en las siguientes:

\begin{itemize}
	\item \textbf{Recuperacion de errores}
		
		Cuando hay grupos de procesos que
		realizan acciones atómicas, la detección de un error en un
		proceso necesita de la participación del resto de procesos en la
		recuperación. Por ejemplo, un fallo hardware puede suponer que
		el proceso nunca termine su ejecución prevista porque las
		condiciones de comienzo ya no se cumplen; el proceso puede que
		nunca alcance su punto de sondeo. También podría ocurrir un
		fallo de temporización, por lo que no se podría cumplir el plazo
		límite de su servicio. En estas situaciones el proceso debe ser
		informado de que ha sido detectado un error y de que debe
		efectuar alguna forma de recuperación de error lo más
		rápidamente posible.

	\item \textbf{Cambios de modo}
		
		Los sistemas de tiempo real normalmente tienen
		distintos modos de operacion. En ocasiones habrá cambios de
		modo, y estos deben ser gestionados en puntos bien definidos de
		la ejecución del sistema. Sin embargo, en algunas áreas de
		aplicación, los cambios de modo pueden ser esperados pero no
		planificados. En estas situaciones, los procesos deben ser
		informados de forma rápida y segura de que el modo en el que
		estaban operando ha cambiado y deben proceder a un conjunto de
		acciones diferentes.

	\item \textbf{Planificación utilizando computaciones parciales/
		imprecisas}

		Existen muchos algoritmos en los que la precisión de los
		resultados depende del tiempo asignado a los cálculos. Por
		ejemplo, los cálculos numéricos, computaciones estadísticas y
		búsquedas heurísticas, generan una estimación inicial del
		resultado requerido, que despues es mejorada con una mayor
		precisión. En tiempo de ejecución, a cada algoritmo se le asigna
		una cierta cantidad de tiempo, transcurrido el cual el proceso
		debe interrumpirse, de forma que se finaliza el mecanismo de
		refinamiento del resultado.

	\item \textbf{Interrupciones de usuario}
		
		En un entorno interactivo general, los
		usuarios a menudo desean finalizar el procesamiento actual
		porque han detectado una condición de error y desean comenzar de
		nuevo.
\end{itemize}

Una aproximación al manejo asíncrono de notificación es abortar el proceso y
permitir que otro proceso efectúe alguna forma de recuperación. Todos los
sistemas operativos y la mayoría de los lenguajes de programación concurrente
proporcionan esta disponibilidad. Sin embargo, el aborto de un proceso puede ser
costoso y a menudo es una respuesta extrama a muchas condiciones de error.
Consecuentemente, es necesario algún mecanismo de notificación asíncrona.
