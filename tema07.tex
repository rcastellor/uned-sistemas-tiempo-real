\section{Acciones atómicas, Tareas Concurrentes y Fiabilidad}

En este tema se amplian las discusiones iniciales sobre tolerancia a fallos
describiendo con cuánta fiabilidad puede ser programada la cooperación entre
procesos. Para esta discusión es fundamental la noción de acción atómica y las
técnicas de manejo de eventos asíncronos.

Los objetivos del tema pasan por:

\subsection{Acciones atómicas}

Una acción es atómica si las tareas que la llevan a cabo no son conscientes de
las actividades de las primeras, durante el tiempo en el que llevan a cabo la
acción atómica.

\begin{itemize}
	\item Sin comunicación con otras tareas
	\item No detectan cambios de estado externos
	\item Los cambios de estado internos no se comunican hasta finalizar
\end{itemize}

Las acciones atómicas son vistas desde fuera como indivisibles e instantáneas.

\subsection{Acciones atómicas recuperables}

\subsection{Notificación asíncrona}
