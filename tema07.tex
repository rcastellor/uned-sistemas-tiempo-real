\section{Acciones atómicas, Tareas Concurrentes y Fiabilidad}

En este tema se amplian las discusiones iniciales sobre tolerancia a fallos
describiendo con cuánta fiabilidad puede ser programada la cooperación entre
procesos. Para esta discusión es fundamental la noción de acción atómica y las
técnicas de manejo de eventos asíncronos.

Los objetivos del tema pasan por:

\subsection{Acciones atómicas}

Una acción es atómica si las tareas que la llevan a cabo no son conscientes de
las actividades de las primeras, durante el tiempo en el que llevan a cabo la
acción atómica.

\begin{itemize}
	\item Sin comunicación con otras tareas
	\item No detectan cambios de estado externos
	\item Los cambios de estado internos no se comunican hasta finalizar
\end{itemize}

Las acciones atómicas son vistas desde fuera como indivisibles e instantáneas.

Para permitir la descomposición modular de las acciones atomicas se incorpora la
noción de acciones atómicas anidadas. Los procesos comprometidos en una acción
anidada deben ser un subconjunto de los que están involucrados en el nivel
externo de la acción.

\subsection{Acciones atómicas recuperables}

La expresión transacciones atómicas se ha utilizado frecuentemente en el marco
conceptual de los sistemas operativos y las bases de datos. Una transacción
atómica tiene todas las propiedades de una acción atómica, más la
característica adicional de que su ejecución puede tener exito o fallar. Por
fallo se entiende la ocurrencia de un error del que la transacción no puede
recuperarse, por ejemplo, un fallo de procesador. Si falla una unidad atómica,
los componentes del sistema que están siendo manipulados por la acción pueden
quedar en estado inconsistente. Ante un fallo, una transacción atómica
garantiza que los componentes son devueltos a su estado original. Las
transacciones atómicas a veces se conocen como acciones recuperables.

Las dos propiedades distintivas de las transacciones atómicas son:

\begin{itemize}
	\item Atomicidad de fallo, lo que significa que la transacción debe o
		bien ser completada con éxito o (en el caso de fallar) no tener
		efecto.
	\item Atomicidad de sincronización (o aislamiento), lo que significa
		que la transacción es indivisible, en el sentido de que su
		ejecución parcial no puede ser observada por ninguna transacción
		que se esté ejecutando concurrentemente.
\end{itemize}

\subsection{Notificación asíncrona}
