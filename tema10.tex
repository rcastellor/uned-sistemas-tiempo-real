\section{Planificación}

Tras el tema anterior de capacidades de tiempo real, hay que incluir estas
capacidades en la planificación de tareas, introduciéndose también la noción de
prioridad junto con el análisis de la planificabilidad para sistemas con
desalojo basado en prioridad.

Vamos a barajar tres modelos de planificación:

\begin{itemize}
	\item Modelo simple
	\item Ejecutivo cíclico
	\item Planificación basada en procesos
\end{itemize}

\subsection{Modelo simple}

Tenemos un modelo que permite describir algunos esquemas de planificación
estándar:

\begin{itemize}
	\item La aplicación está compuesta por un conjunto fijo de procesos
	\item Los procesos son periódicos, con periodos conocidos
	\item Los procesos son independientes entre sí
	\item El instante crítico es cuando todos los procesos son ejecutados a
		la vez
	\item Las sobrecargas del sistema, tiempos de cambio de contexto y demás
		se suponen con coste cero
	\item Los tiempos límite de los procesos son iguales a sus periodos
	\item Los procesos tienen tiempo de ejecución constante en el peor caso
\end{itemize}

Al considerar que hay independencia entre procesos se puede suponer que en algun
instante de tiempo todos los procesos son ejecutados a la vez, esto representara
la carga máxima para el procesador y se conocerá como \textbf{instante crítico}.

Para definir el modelo de proceso simple definimos las siguientes variables:

\begin{tabular}{|c|l|}
	\hline 
	Notación & Descripción \\ 
	\hline 
	B &  Tiempo de bloqueo del proceso en el peor caso \\ 
	\hline 
	C &  Tiempo de ejecución del proceso en el peor caso (WCET) \\ 
	\hline
	D &  Tiempo límite del proceso \\
	\hline
	I &  Tiempo de interferencia del proceso \\
	\hline
	J &  Fluctuación en la ejecución del proceso \\
	\hline
	N &  Número de procesos en el sistema \\
	\hline
	P &  Prioridad asignada al proceso \\
	\hline
	R &  Tiempo de respuesta del proceso en el peor caso \\
	\hline
	T &  Tiempo mínimo entre ejecuciones del proceso (periodo) \\
	\hline
	Y &  Utilización de cada proceso (C/T) \\
	\hline
	a-z &  Nombre del proceso \\
	\hline
\end{tabular} 


\subsection{Ejecutivo cíclico}

Una forma común de la implementación de sistemas de tiempo real es el uso de un
ejecutivo cíclico.


El ejecutivo cíclico es una tabla de llamadas a procedimientos, donde cada
procedimiento representa parte del código de un proceso. A la tabla completa se
le conoce como \textbf{ciclo principal} y habitualmente consta de cierto número
de \textbf{ciclos secundarios}, cada uno de ellos de duración fija. De esta
forma, por ejemplo, cuatro ciclos secundarios de 25 ms de duración conforman un
ciclo principal de 100 ms. Durante la ejecución, una interrupción de reloj cada
25 ms premite que el planificador realice rondas entre los cuatro ciclos
secundarios.

Existe una tabla de llamadas a procedimiento con un ciclo principal y ciclos
secundarios de duración fija.

Las propiedades del ejecutivo cíclico son:

\begin{itemize}
	\item Los procesos no existen en tiempo de ejecución, solamente existe
		el proceso principal
	\item Los procedimientos comparten un espacio de direcciones, pueden
		compartir datos sin necesidad de protección
	\item Los periodos deben ser múltiplos del tiempo de ciclo secundario
\end{itemize}

El ejecutivo cíclico presenta los siguientes inconvenientes:

\begin{itemize}
	\item Dificultad para incorporar procesos esporádicos
	\item Dificultad para incorporar procesos con periodos grandes, el
		tiempo del ciclo mayor es el único que se puede usar sin
		replanificación
	\item Dificultad para construir el ejecutivo cíclico
	\item Procesos con tiempo notable tendrán que ser divididos en
		procedimientos de tamaño fijo, es propenso a errores
\end{itemize}

Si puede construirse un ejecutivo cíclico no será necesario ningún test de
planificabilidad más. El problema que plantea el ejecutivo cíclico es que a
partir de cierto tamaño se convierte en irresoluble, para sistemas periódicos
simples se mantiene como una estrategia de implementación apropiada.

\subsection{Planificación basada en procesos}

Una alternativa al enfoque del ejecutivo cíclico(la ejecución consiste en una
secuencia de llamadas a procedimientos) es soportar de forma directa la
ejecución de procesos y determinar cual es el proceso que deberá ejecutarse en
cada instante de tiempo mediante uno o más atributos de planificación. Un
proceso está limitado a estar en uno de los posibles estados siguientes:

\begin{itemize}
	\item Ejecutable
	\item Suspendido en espera de un evento temporizado (apropiado para
		procesos periódicos)
	\item Suspendido en espera de un evento no temporizado (apropiado para
		procesos esporádicos)
\end{itemize}

\subsubsection{Alternativas de planificación}

Existe un gran número de aproximaciones distintas a la planificación:
